% Options for packages loaded elsewhere
\PassOptionsToPackage{unicode}{hyperref}
\PassOptionsToPackage{hyphens}{url}
%
\documentclass[
]{article}
\usepackage{lmodern}
\usepackage{amssymb,amsmath}
\usepackage{ifxetex,ifluatex}
\ifnum 0\ifxetex 1\fi\ifluatex 1\fi=0 % if pdftex
  \usepackage[T1]{fontenc}
  \usepackage[utf8]{inputenc}
  \usepackage{textcomp} % provide euro and other symbols
\else % if luatex or xetex
  \usepackage{unicode-math}
  \defaultfontfeatures{Scale=MatchLowercase}
  \defaultfontfeatures[\rmfamily]{Ligatures=TeX,Scale=1}
\fi
% Use upquote if available, for straight quotes in verbatim environments
\IfFileExists{upquote.sty}{\usepackage{upquote}}{}
\IfFileExists{microtype.sty}{% use microtype if available
  \usepackage[]{microtype}
  \UseMicrotypeSet[protrusion]{basicmath} % disable protrusion for tt fonts
}{}
\makeatletter
\@ifundefined{KOMAClassName}{% if non-KOMA class
  \IfFileExists{parskip.sty}{%
    \usepackage{parskip}
  }{% else
    \setlength{\parindent}{0pt}
    \setlength{\parskip}{6pt plus 2pt minus 1pt}}
}{% if KOMA class
  \KOMAoptions{parskip=half}}
\makeatother
\usepackage{xcolor}
\IfFileExists{xurl.sty}{\usepackage{xurl}}{} % add URL line breaks if available
\IfFileExists{bookmark.sty}{\usepackage{bookmark}}{\usepackage{hyperref}}
\hypersetup{
  pdftitle={Multivariate},
  pdfauthor={Xiaoyang Li},
  hidelinks,
  pdfcreator={LaTeX via pandoc}}
\urlstyle{same} % disable monospaced font for URLs
\usepackage[margin=1in]{geometry}
\usepackage{color}
\usepackage{fancyvrb}
\newcommand{\VerbBar}{|}
\newcommand{\VERB}{\Verb[commandchars=\\\{\}]}
\DefineVerbatimEnvironment{Highlighting}{Verbatim}{commandchars=\\\{\}}
% Add ',fontsize=\small' for more characters per line
\usepackage{framed}
\definecolor{shadecolor}{RGB}{248,248,248}
\newenvironment{Shaded}{\begin{snugshade}}{\end{snugshade}}
\newcommand{\AlertTok}[1]{\textcolor[rgb]{0.94,0.16,0.16}{#1}}
\newcommand{\AnnotationTok}[1]{\textcolor[rgb]{0.56,0.35,0.01}{\textbf{\textit{#1}}}}
\newcommand{\AttributeTok}[1]{\textcolor[rgb]{0.77,0.63,0.00}{#1}}
\newcommand{\BaseNTok}[1]{\textcolor[rgb]{0.00,0.00,0.81}{#1}}
\newcommand{\BuiltInTok}[1]{#1}
\newcommand{\CharTok}[1]{\textcolor[rgb]{0.31,0.60,0.02}{#1}}
\newcommand{\CommentTok}[1]{\textcolor[rgb]{0.56,0.35,0.01}{\textit{#1}}}
\newcommand{\CommentVarTok}[1]{\textcolor[rgb]{0.56,0.35,0.01}{\textbf{\textit{#1}}}}
\newcommand{\ConstantTok}[1]{\textcolor[rgb]{0.00,0.00,0.00}{#1}}
\newcommand{\ControlFlowTok}[1]{\textcolor[rgb]{0.13,0.29,0.53}{\textbf{#1}}}
\newcommand{\DataTypeTok}[1]{\textcolor[rgb]{0.13,0.29,0.53}{#1}}
\newcommand{\DecValTok}[1]{\textcolor[rgb]{0.00,0.00,0.81}{#1}}
\newcommand{\DocumentationTok}[1]{\textcolor[rgb]{0.56,0.35,0.01}{\textbf{\textit{#1}}}}
\newcommand{\ErrorTok}[1]{\textcolor[rgb]{0.64,0.00,0.00}{\textbf{#1}}}
\newcommand{\ExtensionTok}[1]{#1}
\newcommand{\FloatTok}[1]{\textcolor[rgb]{0.00,0.00,0.81}{#1}}
\newcommand{\FunctionTok}[1]{\textcolor[rgb]{0.00,0.00,0.00}{#1}}
\newcommand{\ImportTok}[1]{#1}
\newcommand{\InformationTok}[1]{\textcolor[rgb]{0.56,0.35,0.01}{\textbf{\textit{#1}}}}
\newcommand{\KeywordTok}[1]{\textcolor[rgb]{0.13,0.29,0.53}{\textbf{#1}}}
\newcommand{\NormalTok}[1]{#1}
\newcommand{\OperatorTok}[1]{\textcolor[rgb]{0.81,0.36,0.00}{\textbf{#1}}}
\newcommand{\OtherTok}[1]{\textcolor[rgb]{0.56,0.35,0.01}{#1}}
\newcommand{\PreprocessorTok}[1]{\textcolor[rgb]{0.56,0.35,0.01}{\textit{#1}}}
\newcommand{\RegionMarkerTok}[1]{#1}
\newcommand{\SpecialCharTok}[1]{\textcolor[rgb]{0.00,0.00,0.00}{#1}}
\newcommand{\SpecialStringTok}[1]{\textcolor[rgb]{0.31,0.60,0.02}{#1}}
\newcommand{\StringTok}[1]{\textcolor[rgb]{0.31,0.60,0.02}{#1}}
\newcommand{\VariableTok}[1]{\textcolor[rgb]{0.00,0.00,0.00}{#1}}
\newcommand{\VerbatimStringTok}[1]{\textcolor[rgb]{0.31,0.60,0.02}{#1}}
\newcommand{\WarningTok}[1]{\textcolor[rgb]{0.56,0.35,0.01}{\textbf{\textit{#1}}}}
\usepackage{graphicx}
\makeatletter
\def\maxwidth{\ifdim\Gin@nat@width>\linewidth\linewidth\else\Gin@nat@width\fi}
\def\maxheight{\ifdim\Gin@nat@height>\textheight\textheight\else\Gin@nat@height\fi}
\makeatother
% Scale images if necessary, so that they will not overflow the page
% margins by default, and it is still possible to overwrite the defaults
% using explicit options in \includegraphics[width, height, ...]{}
\setkeys{Gin}{width=\maxwidth,height=\maxheight,keepaspectratio}
% Set default figure placement to htbp
\makeatletter
\def\fps@figure{htbp}
\makeatother
\setlength{\emergencystretch}{3em} % prevent overfull lines
\providecommand{\tightlist}{%
  \setlength{\itemsep}{0pt}\setlength{\parskip}{0pt}}
\setcounter{secnumdepth}{-\maxdimen} % remove section numbering
\ifluatex
  \usepackage{selnolig}  % disable illegal ligatures
\fi

\title{Multivariate}
\author{Xiaoyang Li}
\date{1/24/2022}

\begin{document}
\maketitle

\hypertarget{variable-selection}{%
\section{Variable selection}\label{variable-selection}}

\hypertarget{include-392-patients-after-remove-na-in-age_group-gender-provoked-homemed-se_group-benzonum}{%
\subsection{include 392 patients after remove NA in age\_group, gender,
provoked, homemed, se\_group,
benzonum}\label{include-392-patients-after-remove-na-in-age_group-gender-provoked-homemed-se_group-benzonum}}

In this case, I exclude variables related to Benzodiazepine prior to ED
visit and Time of 1st dose of benzodiazepine to time of study drug
administration 0/15/30/45/. These 2 variable include large number of
NAs. In addition, from the bar plot, univariate analysis, and above
variable selection, they have none influence to Failure odds of drug B
and drug C

After adjusting the forward selection threshold, all of three variable
selection method generate the same results

For drug A, they generate formula as

\[Failure \sim age +benzonum\]

\begin{Shaded}
\begin{Highlighting}[]
\FunctionTok{glm}\NormalTok{(Failure }\SpecialCharTok{\textasciitilde{}}\NormalTok{age\_group }\SpecialCharTok{+}\NormalTok{ benzonum, }\AttributeTok{data =}\NormalTok{ data\_vs2A, }\AttributeTok{family =} \StringTok{"binomial"}\NormalTok{) }\SpecialCharTok{\%\textgreater{}\%} \FunctionTok{summary}\NormalTok{()}
\end{Highlighting}
\end{Shaded}

\begin{verbatim}
## 
## Call:
## glm(formula = Failure ~ age_group + benzonum, family = "binomial", 
##     data = data_vs2A)
## 
## Deviance Residuals: 
##     Min       1Q   Median       3Q      Max  
## -1.6683  -1.0657  -0.7091   1.1261   1.7343  
## 
## Coefficients:
##                 Estimate Std. Error z value Pr(>|z|)    
## (Intercept)      -1.2524     0.3250  -3.854 0.000116 ***
## age_groupadult    0.3936     0.4008   0.982 0.326173    
## age_groupsenior   1.3742     0.4908   2.800 0.005107 ** 
## benzonum3~7       0.9839     0.3586   2.743 0.006080 ** 
## ---
## Signif. codes:  0 '***' 0.001 '**' 0.01 '*' 0.05 '.' 0.1 ' ' 1
## 
## (Dispersion parameter for binomial family taken to be 1)
## 
##     Null deviance: 200.88  on 148  degrees of freedom
## Residual deviance: 186.43  on 145  degrees of freedom
## AIC: 194.43
## 
## Number of Fisher Scoring iterations: 4
\end{verbatim}

For drug B, they generate formula as

\[Failure \sim provoked + se\ group\]

\begin{Shaded}
\begin{Highlighting}[]
\FunctionTok{glm}\NormalTok{(Failure }\SpecialCharTok{\textasciitilde{}}\NormalTok{provoked }\SpecialCharTok{+}\NormalTok{ se\_group, }\AttributeTok{data =}\NormalTok{ data\_vs2B, }\AttributeTok{family =} \StringTok{"binomial"}\NormalTok{) }\SpecialCharTok{\%\textgreater{}\%} \FunctionTok{summary}\NormalTok{()}
\end{Highlighting}
\end{Shaded}

\begin{verbatim}
## 
## Call:
## glm(formula = Failure ~ provoked + se_group, family = "binomial", 
##     data = data_vs2B)
## 
## Deviance Residuals: 
##     Min       1Q   Median       3Q      Max  
## -1.5221  -1.1750  -0.5609   1.1798   1.9633  
## 
## Coefficients:
##                 Estimate Std. Error z value Pr(>|z|)  
## (Intercept)      -0.9830     0.8396  -1.171   0.2417  
## provokedYes      -0.7871     0.4066  -1.936   0.0529 .
## se_group[30,60)   1.7644     0.8391   2.103   0.0355 *
## se_group[60,)     1.3507     0.8273   1.633   0.1025  
## ---
## Signif. codes:  0 '***' 0.001 '**' 0.01 '*' 0.05 '.' 0.1 ' ' 1
## 
## (Dispersion parameter for binomial family taken to be 1)
## 
##     Null deviance: 167.53  on 120  degrees of freedom
## Residual deviance: 157.70  on 117  degrees of freedom
## AIC: 165.7
## 
## Number of Fisher Scoring iterations: 4
\end{verbatim}

For drug C, they generate formula as

\[Failure \sim homemed\] \emph{Provoked is the last removed variable}

\begin{Shaded}
\begin{Highlighting}[]
\FunctionTok{glm}\NormalTok{(Failure }\SpecialCharTok{\textasciitilde{}}\NormalTok{homemed, }\AttributeTok{data =}\NormalTok{ data\_vs2C, }\AttributeTok{family =} \StringTok{"binomial"}\NormalTok{) }\SpecialCharTok{\%\textgreater{}\%} \FunctionTok{summary}\NormalTok{()}
\end{Highlighting}
\end{Shaded}

\begin{verbatim}
## 
## Call:
## glm(formula = Failure ~ homemed, family = "binomial", data = data_vs2C)
## 
## Deviance Residuals: 
##    Min      1Q  Median      3Q     Max  
## -1.220  -0.914  -0.914   1.135   1.466  
## 
## Coefficients:
##             Estimate Std. Error z value Pr(>|z|)  
## (Intercept)   0.1001     0.3166   0.316   0.7519  
## homemedYes   -0.7569     0.3930  -1.926   0.0541 .
## ---
## Signif. codes:  0 '***' 0.001 '**' 0.01 '*' 0.05 '.' 0.1 ' ' 1
## 
## (Dispersion parameter for binomial family taken to be 1)
## 
##     Null deviance: 164.38  on 121  degrees of freedom
## Residual deviance: 160.64  on 120  degrees of freedom
## AIC: 164.64
## 
## Number of Fisher Scoring iterations: 4
\end{verbatim}

\end{document}
